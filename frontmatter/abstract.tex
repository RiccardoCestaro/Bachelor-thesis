%!TEX root = ../dissertation.tex
% the abstract


Il \emph{Google Play Store}, che offre la possibilità a chiunque abbia la licenza di poter pubblicare la propria idea, contiene tutt'ora più di tre milioni di applicazioni.
Sfortunatamente l'analisi per individuare comportamenti malevoli prima della pubblicazione di una nuova applicazione non è sufficiente a garantire una selezione accurata tra software benigno e maligno, portando gli utenti inesperti a scaricare dal \emph{Google Play Store} applicazioni con scopi malevoli. 

Una nuova tecnica che utilizzano gli attaccanti per realizzare software malevolo è la virtualizzazione, un tempo realizzata per poter aggirare le dimensioni massime di un pacchetto.
Grazie ad essa, gli sviluppatori hanno realizzato delle applicazioni che permettono di installare altre applicazioni ospite al loro interno, con tutti i problemi di sicurezza che ne derivano. Degli esempi di motori di virtualizzazione sono \emph{VirtualApp} o \emph{DroidPlugin}. L'idea di questi motori di virtualizzazione è la creazione di un ambiente virtuale superiore al framework di Android utilizzando i \emph{Java Dynamic Proxy}, intercettando e modificando le chiamate alle \emph{API}\footnote{Application Programming Interface}.

Gli utenti inesperti utilizzano queste applicazioni per poter avere due istanze della stessa applicazione contemporaneamente. Al giorno d'oggi possedere più di due account di un servizio è ricorrente, ma molte volte non è possibile accederci simultaneamente. Sono pochi, infatti, i servizi che permettono questa funzionalità.
% Inoltre installare due volte la stessa applicazione non è possibile vista la restrizione di un singolo user id (UID) per pacchetto.

I problemi di sicurezza legati alle applicazioni di virtualizzazione sono molteplici, infatti quest'ultime avviano le applicazioni ospite in un layer superiore al sistema operativo virtualizzando l'esecuzione. L'installazione delle applicazioni ospite inoltre è automatica e non viene richiesta nessuna autorizzazione. Il problema più grave riguarda i permessi, infatti, tutti i permessi concessi alla applicazione madre, saranno automaticamente concessi alle applicazioni ospite. Quindi ogni applicazione ospite potrà accedere a dati sensibili senza che essa sia stata opportunamente autorizzata a farlo.

Attualmente sono pochi gli studi riguardanti l'individuazione di tecniche per l' identificazione di ambienti virtualizzati, le uniche si concretizzano nelle librerie \emph{DiPrint} e \emph{Anti-Plugin} che risultano non essere sufficienti per bloccare potenziali malware. L'inefficacia delle librerie attualmente esistenti verrà dimostrata attraverso \emph{Màscara}, un malware basato sulla virtualizzazione. Grazie a dei framework di hooking è possibile intercettare le chiamate alle \emph{API} utilizzate dalle librerie attualmente esistenti e cambiarne i valori di ritorno, falsificando i risultati.

Una strada percorribile per creare una nuova tecnica è realizzare una libreria che operi a basso livello e che vada ad analizzare le strutture interne del \emph{Android Runtime (ART) environment} andando a controllare i cambiamenti dei metodi coinvolti con una possibile virtualizzazione.
Per questo è stata realizzata la libreria \emph{Singular} che permette di identificare un ambiente virtualizzato confrontando la complessità dei metodi e il loro tipo di compilazione attraverso il field \emph{hotness\_count}, proponendo un metodo di identificazione più sofisticato e robusto.


