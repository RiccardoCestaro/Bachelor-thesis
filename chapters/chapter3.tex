%!TEX root = ../dissertation.tex
%\begin{savequote}[75mm]
%This is some random quote to start off the chapter.
%\qauthor{Firstname lastname}
%\end{savequote}

\chapter{Librerie di detection esistenti}
\label{chap:lib_esis}

Le librerie esistenti per l'identificazione di ambienti virtualizzati, \emph{DiPrint}\cite{DiPrint} e \emph{Anti-Plugin}\cite{Antiplugin}, utilizzano tecniche di identificazione che operano a livello \emph{Java}.
In questo capitolo verrà dimostrata l'inefficacia delle librerie attualmente esistenti mostrando la loro inaffidabilità a tempo di esecuzione e mettendo in evidenza la facilità di applicare degli \emph{\gls{hookg}} alle \emph{\gls{apig}} per alterarne il loro comportamento. Tutte le tecniche di identificazione esistenti verranno descritte criticamente mostrando il codice utilizzando per aggirarle implementato in \emph{Màscara}.
Tutte le tecniche di identificazione di ambienti virtualizzati sono tracciate per semplicità in Tab. \ref{tab:tracc_id}.

\newpage



\section{Applicazione degli hook}

\subsection*{Hook con java dynamic proxy}

Nel caso in cui si voglia applicare l'\emph{\gls{hookg}} a una chiamata a servizio di \emph{Android}, allora il processo di realizzazione di un \emph{\gls{hookg}} richiede la creazione di tre classi:

\begin{itemize}
    \item \emph{HookHandle} \texttt{class}: contiene tante classi interne quanti sono i metodi a cui vogliamo applicare degli \emph{\gls{hookg}} a una specifica classe. Per ogni classe interna è possibile alterare il comportamento del metodo originale prima e dopo la sua invocazione;
    \item \emph{BinderHook} \texttt{class}:  ritorna un oggetto proxy al servizio di \emph{Android} su cui si vuole applicare un \emph{\gls{hookg}}, oltre a un'instanza della classe \emph{HookHandle} corrispondente;
    \item Classe fake: contiene le dichiarazioni di metodi statici \emph{Class} e \emph{asInterface}. I nomi delle specifiche classi dovrebbero essere gli stessi dell'interfaccia che contiene tutti i metodi su cui applicare gli \emph{\gls{hookg}}.
\end{itemize}

Se si vuole applicare un \emph{\gls{hookg}} in qualsiasi altro oggetto la procedura è molto più complessa, in particolare sarà molto difficile applicare un \emph{\gls{hookg}} con i \emph{Dynamic Proxy} di Java nel caso di oggetti non statici o classi finali o private.
Una soluzione è utilizzare degli \emph{\gls{hookg}} operanti a basso livello con \emph{Whale} \cite{whale}.

\subsection*{Hook con Whale}

Un modo diretto per applicare degli \emph{\gls{hookg}}, in stile \emph{\gls{xposedg}}\glsfirstoccur, è utilizzare \emph{Whale}. 
I metodi per applicare gli \emph{\gls{hookg}} sono quelli di \emph{\gls{xposedg}}:
\begin{itemize}
    \item \emph{findAndHookMethod} \emph{method}: metodo utilizzato per trovare il metodo su cui applicare un \emph{\gls{hookg}} e a cui applicare una \emph{\gls{callbackg}}:
    \begin{itemize}
        \item \emph{XC\_MethodHook} \emph{class}: \emph{\gls{callbackg}} per cambiare il comportamento di un metodo prima o dopo la sua invocazione;
        \item \emph{XC\_MethodReplacement} \emph{class}: \emph{\gls{callbackg}} per sostituire completamente  un metodo.
    \end{itemize}
\end{itemize}



\section{Tracciamento delle tecniche di identificazione}

\begin{table} [H]
\begin{tabular}{l|lll}    \toprule
\emph{Codice}  & Libreria & Descrizione \\\midrule
\row A-PERM-1 & Anti-Plugin & Controllo dei permessi abilitati a runtime \\ 
\row A-PERM-2 & Anti-Plugin & Controllo dei permessi abilitati a runtime \\ 
\row D-PERM-1 & DiPrint & Controllo dei permessi abilitati a runtime \\ 
\row A-NPCK-3 & Anti-Plugin & Controllo della registrazione del nome del package \\ 
\row A-NCMP-4 & Anti-Plugin & Controllo dei nomi dei componenti \\ 
\row A-NCMP-5 & Anti-Plugin & Controllo dei nomi dei componenti \\ 
\row A-USID-6 & Anti-Plugin & Controllo di condivisione di UserID tra processi \\ 
\row D-USID-2 & DiPrint & Controllo di condivisione di UserID tra processi \\ 
\row A-AMEM-7 & Anti-Plugin & Controllo path di salvataggio dati \\ 
\row D-AMEM-3 & DiPrint & Controllo path di installazione \\ 
\row D-APRO-4 & DiPrint & Controllo nella memoria di processo della presenza di più \emph{base.apk} \\ 
\row D-APRO-5 & DiPrint & Controllo nella memoria di processo del path di librerie interne \\ 
\row A-NACS-8 & Anti-Plugin & Controllo del numero dei servizi \\ 
\row A-NACS-9 & Anti-Plugin & Controllo del numero dei componenti \\ 
\row A-SBRD-10 & Anti-Plugin & Controllo di un invio di un broadcast \\ 
\row A-CPRP-11 & Anti-Plugin & Controllo di cambio di proprietà a runtime \\ 
\row A-RRUN-12 & Anti-Plugin & Controllo di residui a runtime \\ 
\row D-STTR-6 & DiPrint & Controllo dello stack trace \\  \bottomrule \hline
\end{tabular}

\caption{Tabella del tracciamento delle tecniche di identificazione di ambienti virtualizzati}
\label{tab:tracc_id}
\end{table}

\newpage


\section{Analisi delle tecniche di detection}
\label{sec:analisi_tecniche}

\subsection*{A-PERM-1}
\label{a-perm-1}


Nel caso in cui un'applicazione condivida lo stesso \emph{\gls{useridg}}, tutti i processi a essa sottostanti ereditano tutti i permessi. Una tecnica di identificazione controlla se nell'applicazione ospite esistono dei permessi che non sono stati dichiarati nel suo \emph{\gls{manifestg}}, ma comunque accessibili. Nel caso in cui questo fosse vero, allora l'applicazione sta venendo virtualizzata.
Il codice utilizzato da \emph{Anti-Plugin} per verificare la condivisione dello stesso \emph{\gls{useridg}} da parte di più applicazioni ospite è illustrato nello Snippet \ref{lst:aperm1}.


\begin{lstlisting}[language = Java , frame = trBL , firstnumber = 1 , escapeinside={(*@}{@*)},
label={lst:aperm1}, caption={Codice di A-PERM-1},captionpos=b]]
private void undeclaredPermissionCheck(Context context, PackageManager pm){
    boolean found_undeclared = false;
    List<String> requestedPerms = getDeclaredPermissions(context, pm);
    List<String> allPerms = getAllPermissions(pm);
    allPerms.removeAll(requestedPerms);
    for (String perm : allPerms) {
        if(ContextCompat.checkSelfPermission(context, perm) == 0) {
            found_undeclared = true;
        }
}
\end{lstlisting}


Tramite il metodo \emph{undeclaredPermissionCheck} \emph{Anti-Plugin} ottiene tutti i permessi non dichiarati nel suo \emph{\gls{manifestg}}, e tramite la funzione \emph{ContextCompat.checkSelfPermission(context, perm)} controlla se è possibile accedere a uno dei permessi. 
Per prima cosa vengono recuperati i permessi dichiarati tramite il metodo \emph{getDeclaredPermessions}, illustrato nello Snippet \ref{lst:aperm2}, e salvati in una lista. Attraverso il metodo \emph{getPackageInfo} è possibile ritornare le informazioni riguardanti l'applicazione ospite installata nel dispositivo, in particolare è possibile ritornare i permessi dichiarati nel \emph{\gls{manifestg}}.
Infine, per ottenere la lista dei permessi non dichiarati viene sottratta la lista dei permessi dichiarati alla lista dei permessi totali disponibili, ottenuti tramite il metodo \emph{getAllPermissions}.


\emph{Màscara dichiara nel \gls{manifestg} gli stessi permessi dell'applicazione vittima, in modo da aggirare questa tecnica di identificazione.}

\newpage

\subsection*{A-PERM-2}
\label{a-perm-2}
Il metodo di \emph{Anti-Plugin} che implementa questa tecnica di identificazione è \emph{getDeclaredPermissions}, illustrato nello Snippet \ref{lst:aperm2}, il quale viene utilizzato anche nella tecnica di identificazione \emph{A-PERM-1}.

\begin{lstlisting}[language = Java , frame = trBL , firstnumber = 1 , escapeinside={(*@}{@*)},
label={lst:aperm2}, caption={Codice di A-PERM-2},captionpos=b]]
private List<String> getDeclaredPermissions(Context ctx, PackageManager pm){
    ArrayList<String> perms = new ArrayList<String>();
    String pkgname = ctx.getApplicationContext().getPackageName();
    try {
       PackageInfo PI = pm.getPackageInfo(pkgname,
         PackageManager.GET_CONFIGURATIONS 
         PackageManager.GET_PERMISSIONS |
         PackageManager.GET_ACTIVITIES |
         PackageManager.GET_SERVICES |
         PackageManager.GET_META_DATA
        );
        if(PI.permissions != null) {
            String perm_str = "";
            for (int i = 0; i < PI.permissions.length; i++) {
                perms.add(PI.permissions[i].toString());
                perm_str += PI.permissions[i].toString()+"\n";
            }
        }
        if(PI.requestedPermissions != null) {
            String perm_str = "";
            for (int i = 0; i < PI.requestedPermissions.length; i++) {
                perms.add(PI.requestedPermissions[i].toString());
                perm_str += PI.requestedPermissions[i].toString()+"\n";
            }
        }
    } catch (Exception e) {}
    return perms;
}
\end{lstlisting}

\emph{Anti-Plugin} verifica i propri permessi dichiarati nel manifest attraverso l'oggetto \emph{PackageInfo}. Nel caso in cui l'applicazione non è installata nel dispositivo, ma viene avviata solo all'interno dell'ambiente virtualizzato, allora viene sollevata un'eccezione dato che il metodo \emph{getPackageInfo} non riesce a ottenere le informazioni riguardanti un pacchetto non installato.

\emph{In Màscara l'applicazione vittima, avviata dentro il contenitore malevolo, è installata anche nel dispositivo.}

\newpage

\subsection*{D-PERM-1}
\label{d-perm-1}
La tecnica di identificazione di ambienti virtualizzati basata sui permessi implementata in \emph{DiPrint} risulta essere meno robusta di quella implementata in \emph{Anti-Plugin}. Infatti verifica solamente il permesso di lettura dei contatti, senza verificare se viene dichiarato o meno nel \emph{\gls{manifestg}} dell'applicazione. Il codice utilizzato da \emph{DiPrint} viene illustrato nello Snippet \ref{lst:dperm1}.

\begin{lstlisting}[language = Java , frame = trBL , firstnumber = 1 , escapeinside={(*@}{@*)},
label={lst:dperm1}, caption={Codice di D-PERM-1},captionpos=b]]
public String hasReadContactsPermission() {
    String res = "";
    int perm = checkCallingOrSelfPermission("android.permission.READ_CONTACTS");
    if (perm == PackageManager.PERMISSION_GRANTED) {
        res = "virtualization";
    } else {
        res = "real";
    }
    String res2;
    try {
       Cursor cursor = getContentResolver().query(ContactsContract.Contacts.CONTENT_URI,
                null, null, null, null);
    } catch (Exception e) {
        res2 = "real";
    }
    res2 = "virtualization";
    [..]
}
\end{lstlisting}

\emph{DiPrint} prova ad accedere ai contatti tramite l'\emph{API} \emph{checkCallingOrSelfPermission} verificando il permesso \emph{android.permission.READ\_CONTACTS}.
Subito dopo cerca di accedere ai contatti direttamente tramite la query \emph{ContactsContract.Contacts.CONTENT\_URI} e nel caso in cui sia negato l'accesso viene sollevata un'eccezione.
Nel caso in cui uno dei due tentativi vada a buon fine, viene segnalata una possibile virtualizzazione.

\emph{Màscara dichiara nel \gls{manifestg} gli stessi permessi dell'applicazione vittima, in modo da aggirare questa tipologia di detection.}

%SOLUZIONE:
%Un modo facile e veloce di aggirare questa tecnica di detection è dichiarare nel manifest della guest app gli %stessi permessi dichiarati nel manifest della host app.
%Questa tecnica non funziona nel caso in cui i permessi dichiarati nel manifest della host app sono minori o %uguali ai permessi dichiarati nella guest app.

\newpage

\subsection*{A-NPCK-3}
\label{a-npck-3}

Un'applicazione ospite potrebbe essere avviata anche se non appartiene al sistema operativo. Questa tecnica consiste nel controllare se l'applicazione esista nel dispositivo. \emph{Anti-Plugin} la implementa tramite il metodo \emph{getCurrentAppInfo}, illustrato nello Snippet \ref{lst:anpck3}, e confronta se nella lista ritornata esiste il pacchetto dell'applicazione. 

\begin{lstlisting}[language = Java , frame = trBL , firstnumber = 1 , escapeinside={(*@}{@*)},
label={lst:anpck3}, caption={Codice di A-NPCK-3},captionpos=b]]
private PackageInfo getCurrrentAppInfo(PackageManager pm, String pkgName){
        List<ApplicationInfo> packages = pm.getInstalledApplications(PackageManager.GET_META_DATA);
        for (ApplicationInfo applicationInfo : packages) {
            if(applicationInfo.packageName.equals(pkgName)) {
                try {
                    PackageInfo packageInfo = pm.getPackageInfo(pkgName, PackageManager.GET_PERMISSIONS);
                    String[] requestedPermissions = packageInfo.requestedPermissions;
                    if(requestedPermissions != null) {
                        for (int i = 0; i < requestedPermissions.length; i++) {
                            Log.d("Anti", requestedPermissions[i]);
                        }
                    }
                }  catch (PackageManager.NameNotFoundException e) {
                    e.printStackTrace();
                }catch (Exception e) {}
            }
        }
        return null;
    }


\end{lstlisting}

Attraverso il metodo \emph{getInstalledApplications} \emph{Anti-Plugin} ritorna la lista dei pacchetti installati nel sistema operativo. Successivamente verifica se all'interno della lista esiste l' applicazione tramite il metodo \emph{getPackageInfo} e in caso negativo viene sollevata un'eccezione e segnalata una virtualizzazione.

\emph{In Màscara l'applicazione vittima, avviata dentro il contenitore malevolo, è installata anche nel dispositivo.}

\newpage

\subsection*{A-NCMP-4}
\label{a-ncmp-4}

Un malware virtualizzato potrebbe contenere delle \emph{\gls{activityg}}. Una tecnica di identificazione controlla se sono presenti delle \emph{\gls{activityg}} sconosciute all'interno un'applicazione contenitore e in caso positivo segnala una possibile virtualizzazione. Il codice per implementare questa tecnica è illustrato nello Snippet \ref{lst:ancmp4}.

\begin{lstlisting}[language = Java , frame = trBL , firstnumber = 1 , escapeinside={(*@}{@*)},
label={lst:ancmp4}, caption={Codice di A-NCMP-4},captionpos=b]]
protected void getCurrentProcessInfo3(Context context) {
    final ActivityManager activityManager = (ActivityManager) context.getSystemService(Context.ACTIVITY_SERVICE);
    final List<ActivityManager.RunningTaskInfo> recentTasks = activityManager.getRunningTasks(Integer.MAX_VALUE);
    String str = "";
    for (int i = 0; i < recentTasks.size(); i++)
    {
        str += "\n\tApplication executed : " +recentTasks.get(i).baseActivity.toShortString()+ "\t\t ID: "+recentTasks.get(i).id+"";
    }
}

\end{lstlisting}

\emph{Anti-Plugin} attraverso le \emph{API} \emph{getRecentTasks} e \emph{getRunningTrasks} ritorna la lista dei \emph{\gls{taskg}}\glsfirstoccurspace\\ recentemente avviati dall'utente o in esecuzione. Dalle \emph{\gls{apig} 21} queste chiamate sono state deprecate e possono ritornare solo i \emph{\gls{taskg}} appartenenti all' applicazione chiamante. In un ambiente virtualizzato \emph{Anti-Plugin} può identificare se esistono altri \emph{\gls{taskg}} appartenenti a altre applicazioni.
Infatti nel caso siano maggiori rispetto a quelli previsti viene segnalata una possibile virtualizzazione.

\emph{In Màscara, oltre all'applicazione vittima, non ci sono activity appartenenti a pacchetti malevoli, che sono formati da soli servizi.}

%DroidPlugin utilizza degli stub per dichiarare i componenti predefiniti nel manifest. Tramite il metodo %getRunningServices della classe ActivityManager è possibile ritornare le informazioni dei componenti, come i %nomi reali degli stub.

\newpage
\subsection*{A-NCMP-5}
\label{a-ncmp-5}

\emph{Anti-Plugin} attraverso la \emph{\gls{apig}} \emph{getRunningServices} ritorna la lista dei servizi in esecuzione. In un ambiente virtualizzato la suddetta \emph{\gls{apig}} dovrebbe individuare tutti i servizi in esecuzione compresi quelli riguardanti i pacchetti malevoli. Il codice implementato in \emph{Anti-Plugin} è illustrato nello Snippet \ref{lst:ancmp5}.

\begin{lstlisting}[language = Java , frame = trBL , firstnumber = 1 , escapeinside={(*@}{@*)},
label={lst:ancmp5}, caption={Codice di A-NCMP-5},captionpos=b]]
public boolean ScanServiceName(ActivityManager manager, String service_name){
    boolean isPlugin = true;
    List<ActivityManager.RunningServiceInfo> serviceList = manager.getRunningServices(100);
    for (Iterator<ActivityManager.RunningServiceInfo> iterator = serviceList.iterator(); iterator.hasNext();) {
        ActivityManager.RunningServiceInfo serviceInfo = iterator.next();
        if(!serviceInfo.service.toString().contains("com.google")
                && !serviceInfo.service.toString().contains("com.android")
                && !serviceInfo.service.toString().contains("android.hardware")) {
        }
        if(serviceInfo.service.toString().contains(service_name)){
            isPlugin = false;
        }
    }
    phoneback(ctx, "result", "", ""+isPlugin,"ScanServices");
    return isPlugin;
}

\end{lstlisting}

Il metodo \emph{ScanServiceName} ritorna una lista attraverso il metodo \emph{getRunningServices} e verifica se esistono servizi con firme diverse da quelle standard.

\emph{Con l'applicazione di un \gls{hookg} attraverso i java Dynamic Proxy Màscara riesce a modificare il valore di ritorno della \gls{apig} getRunningServices in modo da nascondere i servizi malevoli.}

% Mostrare come si applica l'hook

\newpage




\subsection*{A-USID-6}
\label{a-usid-6}

Tutti i processi avviati tramite l'applicazione contenitore condividono lo stesso \emph{\gls{useridg}}. Se la libreria di identificazione di ambienti virtualizzati individua nomi di processi sospetti, allora è probabile che stia avvedendo una virtualizzazione.
Il metodo utilizzato da \emph{Anti-Plugin} per verificare se esistono processi con lo stesso \emph{\gls{useridg}} è \emph{checkUIDProcess} ed è illustrato nello Snippet \ref{lst:ausid6}.

\begin{lstlisting}[language = Java , frame = trBL , firstnumber = 1 , escapeinside={(*@}{@*)},
label={lst:ausid6}, caption={Codice di A-USID-6},captionpos=b]]
protected void checkUIDProcess(Context context, ActivityManager am,  String pkgName){
    int pid = android.os.Process.myPid();
    List<String> unknown_proc = new ArrayList<>();
    for (ActivityManager.RunningAppProcessInfo appProcess : am.getRunningAppProcesses()){
        if(!appProcess.processName.contains(pkgName)){
            unknown_proc.add(appProcess.uid+"_"+appProcess.pid+"_"+appProcess.processName);
        }
    }
    phoneback(context, "result", "", ""+(unknown_proc.size() > 0), "checkUIDProcess");
}
\end{lstlisting}

\emph{Anti-Plugin} attraverso l'API \emph{getRunningAppProcesses} ritorna la lista dei processi dell' applicazione attualmente in esecuzione, e verifica se esistono dei processi con dei nomi sospetti, diversi da quelli aspettati dall'applicazione. Salva tutti i processi sospetti in un array \emph{unknown\_proc} e li ritorna.

\emph{Màscara utilizza i Dynamic Proxy di Java per applicare un \gls{hookg} al valore di ritorno dell'API
getRunningAppProcesses in modo da mascherare i nomi dei processi. Infatti tutti i processi condividono lo stesso nome dell'applicazione vittima.}
Il codice malevolo aggiunto in \emph{Màscara} per poter aggirare questa tecnica di identificazione di ambienti virtualizzati è illustrato nello Snippet \ref{lst:ausid6hook}.

\newpage

\begin{lstlisting}[language = Java , frame = trBL , firstnumber = 1 , escapeinside={(*@}{@*)},
label={lst:ausid6hook}, caption={Codice per l'hooking di A-USID-6},captionpos=b]]
@Override
protected void afterInvoke(Object receiver, Method method, Object[] args, Object invokeResult) throws Throwable {
    System.out.println("getRunningAppProcesses:afterInvoke");
        [..]
            for (Object info : infos) {
                if (info instanceof ActivityManager.RunningAppProcessInfo) {
                    ActivityManager.RunningAppProcessInfo myinfo = (ActivityManager.RunningAppProcessInfo) info;
                    if (myinfo.uid != android.os.Process.myUid()) {
                        continue;
                    }
                    List<String> pkgs = PluginManager.getInstance().getPackageNameByPid(myinfo.pid);
                    String processname = PluginManager.getInstance().getProcessNameByPid(myinfo.pid);
                    System.out.println("HOOKPROCESSNAME  " + processname);
                    if(processname.equals("com.panw.lab.blackhatdemo"))
                        myinfo.processName = "com.panw.lab.blackhatdemo";
                    if (pkgs != null && pkgs.size() > 0) {
                        ArrayList<String> ls = new ArrayList<String>();
                        if (myinfo.pkgList != null) {
                            for (String s : myinfo.pkgList) {
                                if (!ls.contains(s)) {
                                    ls.add(s);
                                }
                            }
                        }
                        for (String s : pkgs) {
                            if (!ls.contains(s)) {
                                ls.add(s);
                            }
                        }
                        myinfo.pkgList = ls.toArray(new String[ls.size()]);
            [..]
\end{lstlisting}

Attraverso il metodo \emph{AfterInvoke} per prima cosa vengono iterati i pacchetti del metodo \emph{getRunningAppProcesses} in modo da trovare i processi riguardanti l'applicazione su cui stiamo operando. Per fare questo \emph{Màscara} controlla per ogni pacchetto, tramite il metodo \emph{RunningAppProcessInfo}, se il suo \emph{\gls{useridg}} corrisponde con quello corrente, tramite il metodo \emph{Process.myUid}.

Successivamente \emph{Màscara} recupera per prima cosa il nome del pacchetto tramite l'\emph{API} \emph{getPackageNameById} e poi tutti i processi sottostanti al pacchetto con l'\emph{API} \emph{getProcessNameByPid}. Infine \emph{Màscara} imposta il nome del processo uguale a tutti i processi sottostanti, in modo da nascondere nomi di processi sospetti. 

\emph{Anti-Plugin in questo modo non riuscirà a individuare un nome di processo che riesca a dimostrare una virtualizzazione.}

\subsection*{D-USID-2}
\label{d-usid-2}

Nella libreria \emph{DiPrint} la tecnica riguardante la condivisione di \emph{\gls{useridg}} tra applicazioni ospite viene implementata in modo più robusto.
Il codice che viene utilizzato è illustrato nello Snippet \ref{lst:dusid2}.

\begin{lstlisting}[language = Java , frame = trBL , firstnumber = 1 , escapeinside={(*@}{@*)},
label={lst:dusid2}, caption={Codice di D-USID-2},captionpos=b]]
public static String runShell(String cmd) {
        String res = "real";
        int count = 0;
        Runtime mRuntime = Runtime.getRuntime();
        try {
            Process mProcess = mRuntime.exec(cmd);
            [..]
            while ((currentLine = br.readLine()) != null) {
                if (currentLine.contains("com.example.lu.diprint")) {
                    uid = currentLine.split("   ")[0];
                    break;
                }
            }
            Process mProcess1 = mRuntime.exec(cmd);
            [..]
            while ((findline = br1.readLine()) != null) {
                if (findline.contains(uid) && !findline.contains("com.example.lu.diprint")&& !findline.contains("R ps")) {
                    res = "virtualization";
                    suspiciousproc = suspiciousproc + "\n" + findline;
        [..]
    }

\end{lstlisting}

\emph{DiPrint} utilizza il metodo \emph{exec} della classe \emph{Runtime} per avviare il comando \emph{ps} e ritornare la lista dei processi attualmente in esecuzione. 
Successivamente ritorna l'\emph{\gls{useridg}} dell'applicazione attualmente in esecuzione dalla riga che viene ritornata, contenente il nome del pacchetto. Infine verifica se esiste una riga, ritornata dal metodo \emph{exec}, che contiene lo stesso \emph{\gls{useridg}} ma con un nome di processo sospetto.
In caso positivo viene segnalata una virtualizzazione.

\emph{In Màscara ogni tentativo di accedere al comando ps è stato negato e sostituito dal comando ls, applicando un hook tramite la libreria Whale.}
Il codice malevolo aggiunto a \emph{Màscara} è illustrato nello Snippet \ref{lst:dusid2hook}.

\begin{lstlisting}[language = Java , frame = trBL , firstnumber = 1 , escapeinside={(*@}{@*)},
label={lst:dusid2hook}, caption={Codice per l'hooking di D-USID-2},captionpos=b]]
private void hookRuntimeExec(ClassLoader classLoader) {
    XposedHelpers.findAndHookMethod(Runtime.class, "exec", String.class,
            new XC_MethodHook() {
                @Override
                protected void afterHookedMethod(MethodHookParam param) throws Throwable {
                    android.util.Log.e("wind", "wind -- afterHookedMethod exec! para = " + param.args[0]);
                    param.setResult(Runtime.getRuntime().exec("ls",null,null));
                }
            });
    }

\end{lstlisting}

Semplicemente, all'interno del metodo \emph{afterHookedMethod} viene modificato il valore di ritorno tramite il metodo \emph{setResult} e richiamando \emph{exec} con il comando \emph{ls}.
Per evitare di cadere in un loop infinito di \emph{\gls{hookg}} viene utilizzato il metodo \emph{exec} a tre parametri.

\emph{DiPrint in questo modo non riuscirà ad individuare un nome di processo che riesca a dimostrare una virtualizzazione.}

\newpage

\subsection*{A-AMEM-7}
\label{a-amen-7}

Un'applicazione ospite, solitamente, condivide il proprio spazio in memoria con le altre applicazioni ospite all'interno della cartella riservata all'applicazione contenitore. Infatti i path utilizzati dalle applicazioni ospite sono diversi dai path utilizzati dalle applicazioni installate nativamente in \emph{Android}. \emph{Anti-Plugin} verifica se i path delle applicazioni corrispondono alla loro posizione originale. 
Il codice utilizzato da \emph{Anti-Plugin} per implementare questa tecnica di identificazione è illustrato nello Snippet \ref{lst:aamem7}.

\begin{lstlisting}[language = Java , frame = trBL , firstnumber = 1 , escapeinside={(*@}{@*)},
label={lst:aamem7}, caption={Codice di A-AMEM-7},captionpos=b]]
protected void checkAppRuntimeDir(Context context, PackageManager pm, String pkgName) {
    try {
        ApplicationInfo ai = pm.getApplicationInfo(pkgName, PackageManager.GET_META_DATA | PackageManager.GET_SHARED_LIBRARY_FILES);
        boolean dataDir_wrong = !ai.dataDir.startsWith("/data/user/0/" + pkgName);
        boolean srcDir_wrong = !ai.sourceDir.startsWith("/data/app/" + pkgName);
        boolean pSrcDir_wrong = !ai.publicSourceDir.startsWith("/data/app/" + pkgName);
        phoneback(context, "result", "", ""+(dataDir_wrong | srcDir_wrong | pSrcDir_wrong), "checkAppRuntimeDir");
    } catch (PackageManager.NameNotFoundException e) {
        e.printStackTrace();
    }
}
\end{lstlisting}

Attraverso il metodo \emph{getApplicationInfo} della classe \emph{PackageManager} è possibile ritornare una serie di informazioni riguardanti l'applicazione, come ad esempio i path delle directory.
\emph{Anti-Plugin} verifica se i path \emph{dataDir, sourceDir} e \emph{publicSourceDir} sono sospetti, e in caso positivo segnala una possibile virtualizzazione.

\emph{In Màscara applicando un \gls{hookg} dopo la chiamata dell'API getApplicationInfo è possibile modificare i path dataDir, sourceDir e publicSourceDir in modo che corrispondano ai path dell'applicazione non virtualizzata.} Il codice malevolo aggiunto a \emph{Màscara} per aggirare questa tecnica di detection è illustrato nello Snippet \ref{lst:aamem7hook}.

\newpage

\begin{lstlisting}[language = Java , frame = trBL , firstnumber = 1 , escapeinside={(*@}{@*)},
label={lst:aamem7hook}, caption={Codice per l'hooking di A-AMEM-7},captionpos=b]]
@Override
protected void afterInvoke(Object receiver, Method method, Object[] args, Object invokeResult) throws Throwable {
[..]
if (packageName != null && packageName.equals("com.panw.lab.blackhatdemo") && flags != null) {
    ApplicationInfo info = PluginManager.getInstance().getApplicationInfo(packageName, flags);
    if (info != null) {
        System.out.println("info" + "  " + packageName + "  " + flags);
        if(info.dataDir != null) {
            System.out.println("info" + "  " + info.dataDir + "  " );
            info.dataDir = "/data/data/com.panw.lab.blackhatdemo";
            System.out.println("info" + "  " + info.dataDir + "  ");
        }
        if(info.sourceDir != null) {
            System.out.println("info" + "  " + info.sourceDir + "  " );
           info.sourceDir = "/data/app/com.panw.lab.blackhatdemo-1/base.apk";
            System.out.println("info" + "  " + info.sourceDir + "  ");
        }
        if(info.publicSourceDir != null) {
            System.out.println("info" + "  " + info.publicSourceDir + "  " );
            info.publicSourceDir = "/data/app/com.panw.lab.blackhatdemo-1/base.apk";
            System.out.println("info" + "  " + info.publicSourceDir + "  ");
        }
    }
}
[..]
}
\end{lstlisting}

All'interno del metodo \emph{afterInvoke}, nel caso il package in questione sia la libreria di detection \emph{Anti-Plugin}, vengono modificati i path \emph{dataDir, sourceDir} e \emph{publicSourceDir} con i path dell'applicazione vittima.
Per farlo \emph{Màscara} recupera l'istanza della classe \emph{ApplicationInfo} dell'applicazione vittima attraverso il metodo \emph{getApplicationInfo}, che contiene le informazioni sui path dell'applicazione ospite.
\emph{DiPrint} esegue dei controlli all'interno della memoria di processo per identificare un possibile ambiente virtualizzato. La memoria di processo è possibile leggerla attraverso \emph{/proc/self/maps}.


\subsection*{D-AMEM-3}
\label{d-amen-3}

Le applicazioni ospite condividono lo spazio di memoria secondaria tra di loro e in particolare sono contenute all'interno dello spazio di memoria secondaria dell'applicazione contenitore.
Nel momento in cui un'applicazione contenitore deve virtualizzare l'applicazione ospite, copia l'\emph{APK} dell'applicazione ospite del suo spazio di memoria secondaria. Se la directory dove è contenuto il file \emph{APK} è diversa da quella convenzionale allora viene segnalata una virtualizzazione.

Il codice implementato in \emph{DiPrint} per questa tecnica è illustrato nello Snippet \ref{lst:damem3}.

\begin{lstlisting}[language = Java , frame = trBL , firstnumber = 1 , escapeinside={(*@}{@*)},
label={lst:damem3}, caption={Codice di D-AMEM-3},captionpos=b]]
public String checkAPKCodeLoadingPath() {
    String res = "";
    apkpath = getPackageCodePath();
    Log.i(TAG, apkpath);
        if ((apkpath.equals("/data/app/com.example.lu.diprint-1/base.apk"))||(apkpath.equals("/data/app/com.example.lu.diprint-2/base.apk"))) {
        res = "real";
    } else {
        res = "virtualization";
    }
    return res;
}

\end{lstlisting}

Attraverso il metodo \emph{getPackageCodePath} \emph{DiPrint} recupera il path della posizione attuale del file \emph{APK}.
Se il file \emph{APK} non è contenuto in una delle due directory imposte da \emph{DiPrint} allora viene segnalata una virtualizzazione.
In \emph{Android 10} questa tecnica segnala in ogni caso una virtualizzazione visto i diversi nomi dati alle cartelle di installazione delle applicazioni.

\emph{Màscara aggira questa tecnica applicando un hook al metodo getPackageCodePath in modo che tutte le informazioni ricevute corrispondano alle informazioni dell'applicazione installata nativamente.}




\newpage
\subsection*{D-APRO-4}
\label{d-apro-4}

Un'applicazione contenitore, per poter avviare un'applicazione ospite, necessita di recuperare i \emph{\gls{dexfileg}} dal suo pacchetto \emph{APK}. Le applicazioni installate in \emph{Android} da un utente vengono salvate nella directory \emph{data/app/[name]/}, contenente anche il file \emph{APK} dell'applicazione rinominato in \emph{base.apk}.
Nel caso in cui l'applicazione ospite sia già installata nel dispositivo da un utente, \emph{DiPrint} controlla nella memoria di processo la presenza nel file \emph{base.apk} \\dell' applicazione contenitore.
In caso positivo segnala una virtualizzazione. Il codice utilizzato in \emph{DiPrint} per implementare questa tecnica è disponibile nello Snippet \ref{lst:dapro4}.

\begin{lstlisting}[language = Java , frame = trBL , firstnumber = 1 , escapeinside={(*@}{@*)},
label={lst:dapro4}, caption={Codice di D-APRO-4},captionpos=b]]
public static String checkHostAPK(String filePath) {
    String res = "";
    try {
        String encoding = "GBK";
        File file = new File(filePath);
        if (file.isFile() && file.exists()) {
            [..]
            while ((lineTxt = bufferedReader.readLine()) != null) {
                if ((lineTxt.contains("base.apk") && !lineTxt.contains("com.example.lu.diprint") )) {
                    hostapkpath = lineTxt;
                    flag = 1;
                }
            }
            if (flag == 0) {
                res = "real";
            } else {
                res = "virtualization";
            }
            read.close();
       [..]
}

\end{lstlisting}

Per accedere alla memoria di processo \emph{DiPrint} crea un nuovo \emph{File} con parametro il path di memoria \emph{/proc/self/maps}.
Al suo interno cerca se è possibile trovare un file \emph{base.apk} la quale directory non corrisponda a quella del pacchetto stesso, ma ad una possibile applicazione contenitore.

\emph{In Màscara ogni tentativo di accesso al file /proc/self/maps viene negato attraverso un hook applicato al costruttore della classe File.} Il codice malevolo aggiunto a \emph{Màscara} per poter aggirare questa tecnica è illustrato nello Snippet \ref{lst:dapro4hook}.

\begin{lstlisting}[language = Java , frame = trBL , firstnumber = 1 , escapeinside={(*@}{@*)},
label={lst:dapro4hook}, caption={Codice per l'hooking di D-APRO-4},captionpos=b]]
private void hookProc(ClassLoader classLoader) {
    XposedHelpers.findAndHookConstructor(File.class,String.class,
        new XC_MethodHook() {
            @Override
            protected void beforeHookedMethod(MethodHookParam param){
                android.util.Log.e("wind", "wind -- beforeHookedMethod File constructor!  para = " + param.args[0]);
                if(param.args[0].equals("/proc/self/maps"))
                    param.args[0] = ".";
            }
        });
}
\end{lstlisting}

Sfruttando l'\emph{\gls{hookg}} in stile \emph{Xposed} offerto da \emph{Whale}, la memoria di processo non è accessibile attraverso il costruttore della classe \emph{File}.
Infatti \emph{Màscara} sostituisce il parametro relativo al path nel caso in cui corrisponda in \emph{/proc/self/maps} in un path vuoto.

\emph{In questo modo DiPrint non è più in grado di verificare la memoria di processo e quindi di identificare una virtualizzazione}.

\subsection*{D-APRO-5}
\label{d-apro-5}

Un'applicazione contenitore, per poter avviare le applicazioni ospite, potrebbe aver bisogno di librerie native. Infatti potrebbero esistere librerie contenute in directory diverse da quelle di default. Il problema di questa tecnica di identificazione di ambienti virtualizzati è la presenza nelle versioni più recenti di \emph{Android} di librerie posizionate in directory non identificate da \emph{DiPrint}.
In ogni caso, per effettuare questo controllo, \emph{DiPrint} verifica la memoria di processo contenuta in \emph{/self/proc/maps}.

Il codice implementato in \emph{DiPrint} per questa tecnica è illustrato nello Snippet \ref{lst:dapro5}. 

\begin{lstlisting}[language = Java , frame = trBL , firstnumber = 1 , escapeinside={(*@}{@*)},
label={lst:dapro5}, caption={Codice di D-APRO-5},captionpos=b]]
public static String checkSuspiciousLib(String filePath) {
    String res = "";
    try {
        String encoding = "GBK";
        File file = new File(filePath);
        if (file.isFile() && file.exists()) {
            [..]
            while ((lineTxt = bufferedReader.readLine()) != null) {
                if (lineTxt.contains(".so") && !lineTxt.contains("/system/lib/") && !lineTxt.contains("libmylibrary") && !lineTxt.contains("/system/vendor/lib/") && !lineTxt.contains("/system/lib64/")&& !lineTxt.contains("/system/vendor/lib64/") ) {
                    libpath.append(lineTxt);
                    flag = 1;
                }
            }
            if (flag == 0) {
                res = "real";
            } else {
                res = "virtualization";
            }
            read.close();
        [..]
}
\end{lstlisting}


Questa tecnica di identificazione controlla la presenza di librerie sospette esterne a \emph{DiPrint} mappate nella memoria di processo contenuta nel path \emph{/proc/self/maps}. 
Parte dal presupposto che tutte le librerie mappate nella memoria del processo delle applicazione ospite siano sotto le directory \emph{/system/lib} o \emph{/system/vendor/lib}.
Uno specifico produttore potrebbe inserire librerie anche in altri path e di conseguenza l’applicazione segnalerebbe sempre una virtualizzazione, anche in caso stia eseguendo in \emph{Android} nativo.
Un esempio è in \emph{Android 10}, dove \emph{DiPrint} segnala una virtualizzazione. Questo poichè ci sono molti path, dove sono contenute librerie, differenti da quelli sopra citati, come \emph{/apex/com.android.runtime/lib64} o \emph{/vendor/lib64} o \emph{/system/product/lib64}.
Non è detto quindi che stia avvenendo una virtualizzazione nel caso in cui ci fossero librerie in path differenti da quelli imposti da \emph{DiPrint}.

\emph{In Màscara applicando un hook al contruttore della classe File è possibile evitare ogni tentativo di accesso a /proc/self/maps, come mostrato nello snippet \ref{lst:dapro4hook}}





\newpage
\subsection*{A-NACS-8}
\label{a-nacs-8}

Ogni componente utilizzato da un'applicazione ospite deve essere dichiarato nel \emph{manifest} dell'applicazione contenitore. Per fare questo solitamente vengono dichiarati dei componenti stub.
\emph{DroidPlugin} di default dichiara solo uno stub \emph{\gls{serviceg}}\glsfirstoccur.
\emph{Anti-Plugin} prova  a lanciare dei \emph{\gls{serviceg}} per provare a sollevare un'eccezione.
Il codice implementato in \emph{Anti-Plugin} per avviare i \emph{\gls{serviceg}} è illustrata nello Snippet \ref{lst:anacs8}.

\begin{lstlisting}[language = Java , frame = trBL , firstnumber = 1 , escapeinside={(*@}{@*)},
label={lst:anacs8}, caption={Codice di A-NACS-8},captionpos=b]]
private boolean checkLocalService(Context context, ActivityManager manager){
    Class myService = DummyService.class;
    String service_name = myService.getName();
    this.startSvc(context, myService);
    boolean serviceNameScan = this.ScanServiceName(manager, service_name);
    return serviceNameScan;
}
private boolean checkRemoteService(Context context, ActivityManager manager){
    Class myService = DummyRemoteService.class;
    String service_name = myService.getName();
    this.startSvc(context, myService);
    boolean serviceNameScan = this.ScanServiceName(manager, service_name);
    return serviceNameScan;
}
private boolean startSvc(Context context, Class myService){
    context.startService(new Intent(context, myService));
    return false;
}
\end{lstlisting}

Attraverso i metodi \emph{checkLocalService} e \emph{checkRemoteService} \emph{Anti-Plugin} crea dei nuovi servizi e prova ad avviarli. Successivamente verifica tramite il metodo \emph{ScanServiceName}, illustrato prima nello snippet \ref{lst:ancmp5}, se il servizio è presente ed è stato avviato.
Se il servizio non risulta attivo, allora viene segnalata una virtualizzazione.

\emph{In Màscara le applicazioni ospite vengono personalizzate appositamente per l'applicazione vittima in modo da dichiarare i suoi stessi componenti nel manifest.}

\newpage
\subsection*{A-NACS-9}
\label{a-nacs-9}

Un'applicazione ospite a runtime condivide informazioni diverse con il sistema operativo rispetto alla stessa applicazione avviata nativamente.
Il codice implementato in \emph{Anti-Plugin} per verificare l'integrità delle informazioni è illustrato nello Snippet \ref{lst:anacs9}.

\begin{lstlisting}[language = Java , frame = trBL , firstnumber = 1 , escapeinside={(*@}{@*)},
label={lst:anacs9}, caption={Codice di A-NACS-9},captionpos=b]]
public boolean listAll(Context context,PackageManager pm){
    String pkgname = context.getApplicationContext().getPackageName();
    Map<String, String> PkgInfo = new HashMap<String, String>();
    Map<String, String> AppInfo = new HashMap<String, String>();
    getCurrrentAppInfo(pm, pkgname);
    try {
        PackageInfo PI = pm.getPackageInfo(pkgname,[..]);
        PkgInfo.put("packageName", PI.packageName.toString());
       [..]
    } catch (Exception e) {
        Log.i("Anti", "Error To Get PackageInfo");
    }try{
        ApplicationInfo AI = pm.getApplicationInfo(pkgname, PackageManager.GET_META_DATA | PackageManager.GET_SHARED_LIBRARY_FILES);
        AppInfo.put("backupAgentName", AI.backupAgentName.toString());
        [..]
    }
    return false;
}
private PackageInfo getCurrrentAppInfo(PackageManager pm, String pkgName){
    List<ApplicationInfo> packages = pm.getInstalledApplications(PackageManager.GET_META_DATA);
    for (ApplicationInfo applicationInfo : packages) {
        if(applicationInfo.packageName.equals(pkgName)) {
            try {
                PackageInfo packageInfo = pm.getPackageInfo(pkgName, PackageManager.GET_PERMISSIONS);
                String[] requestedPermissions = packageInfo.requestedPermissions;
                if(requestedPermissions != null) {
                    [..]
}
\end{lstlisting}

Tramite il metodo \emph{getPackageInfo} è possibile ritornare le informazioni riguardanti un pacchetto a runtime. Il metodo \emph{listAll} di \emph{Anti-Plugin} si occupa di recuperare il nome del pacchetto dal contesto dell' applicazione e creare una lista di informazioni riguardante l' applicazione che sta eseguendo. Il metodo \emph{getCurrentAppInfo}, invece, recupera il nome del pacchetto direttamente dalla lista di applicazioni installate nel sistema operativo. Se ci sono incongruenze tra le informazioni recuperate tra i due metodi allora è probabile che stia avvedendo una virtualizzazione. 

\emph{In Màscara le informazioni ritornate tramite l'oggetto PackageInfo corrispondono a quelle dell'applicazione vittima reale.}


\subsection*{A-SBRD-10}
\label{a-sbrd-10}

Un'applicazione virtualizzata non è in grado di disconnettere i receiver. Questa tecnica di identificazione prova a verificare se è possibile disconnettere un receiver e in caso negativo viene segnalata una virtualizzazione.
Il codice implementato in \emph{Anti-Plugin} è illustrato nello Snippet \ref{lst:asbrd10}.

\begin{lstlisting}[language = Java , frame = trBL , firstnumber = 1 , escapeinside={(*@}{@*)},
label={lst:asbrd10}, caption={Codice di A-SBRD-10},captionpos=b]]
public void checkUnregisteredFilter(Context ctx) {
    rhelper.unregisterAction_all();
    Intent static_intent = new Intent(staticACTION);
    ctx.sendBroadcast(static_intent);
    SystemClock.sleep(5000);
    phoneback(ctx, "c2", "DummyReceiver-" + staticACTION, "1", "checkUnregisteredFilter");
}
\end{lstlisting}

La tecnica viene eseguita con il metodo \emph{checkUnregisteredFilter} all’interno della classe \emph{BroadCastCheck}. 
Il metodo disconnette tutti i receiver a runtime, e prova a mandare un broadcast. Se il broadcast non va a buon fine viene sollevata un'eccezione.

\emph{Anti-Plugin non riesce a identificare Màscara attraverso questa tecnica di identificazione, sembra non funzionare.}

\newpage
\subsection*{A-CPRP-11}
\label{a-cprp-11}

In generale gli ambienti virtualizzati, in particolare quelli che utilizzano \emph{DroidPlugin}, non sono in grado di cambiare le proprietà dei componenti a runtime. Questa tecnica di identificazione prova a cambiare delle proprietà ai componenti e in caso di fallimento segnala una virtualizzazione.
Il codice implementato in \emph{Anti-Plugin} per questa tecnica è illustrato nello Snippet \ref{lst:acprp11}.

\begin{lstlisting}[language = Java , frame = trBL , firstnumber = 1 , escapeinside={(*@}{@*)},
label={lst:acprp11}, caption={Codice di A-CPRP-11},captionpos=b]]
public void checkEnabledComp(Context ctx) { 
    ctx.getPackageManager().setComponentEnabledSetting(
            new ComponentName(ctx, receiver_name),
            PackageManager.COMPONENT_ENABLED_STATE_ENABLED,
            PackageManager.DONT_KILL_APP);
    Intent static_intent = new Intent(staticACTION);
    ctx.sendBroadcast(static_intent);
    SystemClock.sleep(5000);
    phoneback(ctx, "c2", "DummyReceiver-" + staticACTION, "1", "checkEnabledComp");
    }
\end{lstlisting}

 Utilizzando le API \emph{setComponentEnabledSetting} si può provare a vedere se un componente può essere abilitato. Se non è possibile abilitarlo viene sollevata un'eccezione. É il metodo \emph{checkEnabledComp} della classe \emph{BroadcastCheck} che si occupa di questo. 
 
\emph{Anti-Plugin non riesce a identificare Màscara attraverso questa tecnica di identificazione, sembra non funzionare.}



\subsection*{A-RRUN-12}
\label{a-rrun-12}

Come spiegato nel capitolo precedente, tutte le applicazioni ospite condividono le stesse informazioni interne. Anche quando un'applicazione ospite non sta eseguendo, certe volte lascia dei residui che possono essere la prova di una possibile virtualizzazione.


\emph{Anti-Plugin} potrebbe condividere l'ambiente di virtualizzazione con altre applicazioni. Se qualche applicazione all'interno dello stesso ambiente ha dei componenti nativi, come una \emph{WebView}, condivide allora dei dati interni con le applicazioni che usano lo stesso componente.

\emph{In Màscara l'apk malevolo non contiene componenti nativi.}

\newpage

\subsection*{D-STTR-6}
\label{d-sttr-6}

In un ambiente virtualizzato il lancio di un'eccezione dovrebbe creare uno \emph{stacktrace} che permetta di vedere i metodi utilizzati per l'avvio dell'applicazione ospite.
Il codice implementato in \emph{DiPrint} per controllare lo \emph{stacktrace} è illustrato nello Snippet \ref{lst:dsttr6}.

\begin{lstlisting}[language = Java , frame = trBL , firstnumber = 1 , escapeinside={(*@}{@*)},
label={lst:dsttr6}, caption={Codice di D-STTR-6},captionpos=b]]
public String getStatckTrace() {
    String res = "...";
    int count =0;
    try {
        throw new Exception("blah");
    }catch (Exception e) {
        for(StackTraceElement stackTraceElement : e.getStackTrace()) {
            String stacktracetmp = "";
            Log.i(TAG, "call stack: " + stackTraceElement.getClassName() + "->" + stackTraceElement.getMethodName());
            stacktracetmp = stackTraceElement.getClassName() + "->" + stackTraceElement.getMethodName();
            System.out.println("stack--"+stacktracetmp);
            stacktrace = stacktrace + "\n" + stacktracetmp;
            if(stacktracetmp.contains("callActivityOnCreate")){
                count = count +1;
            }
        }
       [..]
}

\end{lstlisting}

Secondo la libreria \emph{DiPrint} lo \emph{stacktrace} ritornato da un lancio di un \emph{throw} dovrebbe ritornare il percorso di crezione dell'\emph{\gls{activityg}}, e quindi dovrebbe passare due volte in \emph{callActivityOnCreate}.

\emph{Purtroppo questa tecnica non ha mai funzionato, lo stacktrace ritornato è identico, sia in ambiente nativo che in ambiente virtualizzato.}

\newpage

\section{Inefficacia}

Le tecniche implementate da \emph{DiPrint} e \emph{Anti-Plugin} non sono solo facilmente aggirabili ma sono anche inefficaci. Infatti sono stati effettuati molti test e i risultati non sono convincenti.

La libreria \emph{DiPrint} ha dimostrato di non essere in grado di rilevare un ambiente nativo, infatti ben due delle 6 tecniche implementate rilevano una virtualizzazione con risultato falso positivo (D-AMEM-3 e D-APRO-5).
Vale la stessa cosa anche per \emph{Anti-Plugin} che rileva degli ambienti virtualizzati in \emph{Android} nativo con la tecnica \emph{A-AMEM-7}.
Le due librerie quindi non riescono a identificare un ambiente che non sia virtualizzato, compromettendo la veridicità dei risultati.
Inoltre solo \emph{A-NCMP-5} e \emph{D-APRO-4} individuano sempre una virtualizzazione.
Infatti solo \emph{A-NCMP-5} e \emph{D-APRO-4} forniscono dei risultati soddisfacenti, riuscendo a identificare tutti gli ambienti di virtualizzazione senza falsi positivi.

In Tab. \ref{tab:antidiprint_device_tests} vengono mostrati i risultati dei test eseguiti su \emph{Android 8.1}, garantendo la compatibilità a tutti gli ambienti di virtualizzazione, per mostrare l'inefficacia delle tecniche implementate. 
I risultati nella tabella sono stati identificati con \emph{false} nel caso in cui l'ambiente rilevato sia nativo, mentre \emph{true} se l'ambiente rilevato è virtualizzato.

Inoltre si può vedere come la libreria \emph{Anti-Plugin} sia fortemente specializzata nell' individuare ambienti virtualizzati che utilizzano \emph{DroidPlugin} ma non riesce a individuare con successo altri ambienti virtualizzati.

Infine, nell'ultima colonna, viene mostrato come \emph{Màscara} riesca ad aggirare tutte le tecniche di detection attraverso gli \emph{\gls{hookg}} applicati e spiegati nella sezione precedente.
\newpage



\begin{table} [H]

\makebox[1 \textwidth][c]{       %centering table
\resizebox{1.2 \textwidth}{!}{   %resize table

\begin{tabular}{l|lllllllllll}    \toprule
\emph{Codice}  & Native & A1 & A2 & A3 & A4 & A5 & A6 & A7 & A8 & DroidPlugin & Màscara \\\midrule
\row A-PERM-1 & false & false & true & true & false & false & true & false & false & true & false\\ 
\row D-PERM-1 & false & false & true & true & false & false & true & false & false & true & false \\ 
\row A-PERM-2 & false & false & false & false & false & false & false & false & false & true & false \\ 
\row A-NPCK-3 & false & false & false & false & false & false & false & false & false & true & false \\ 
\row A-NCMP-4 & false & false & false & false & false & false & false & false & false & true & false \\ 
\row A-NCMP-5 & false & true & true & true & true & true & true & true & true & true & false \\ 
\row A-USID-6 & false & false & false & false & false & false & false & false & false & true & false \\ 
\row D-USID-2 & false & false & false & false & false & false & false & false & false & true & false \\ 
\row A-AMEM-7 & true & true & true & true & true & true & true & true & true & true & false \\ 
\row D-AMEM-3 & true & true & true & true & true & true & true & true & true & true & false \\ 
\row D-APRO-4 & false & true & true & true & true & true & true & true & true & true & false \\ 
\row D-APRO-5 & true & true & true & true & true & true & true & true & true & true & false \\ 
\row A-NACS-8 & false & false & false & false & false & false & false & false & false & false & false \\ 
\row A-NACS-9 & false & false & false & false & false & false & false & false & false & false & false \\ 
\row A-SBRD-10 & false & false & false & false & false & false & false & false & false & true & false  \\ 
\row A-CPRP-11 & false & false & false & false & false & false & false & false & false & true & false  \\ 
\row A-RRUN-12 & false & false & false & false & false & false & false & false & false & false & false  \\  
\row D-STTR-6 & false & false & false & false & false & false & false & false & false & false & false  \\   \bottomrule \hline
\end{tabular}


}
}

\caption[Tabella dei risultati di identificazione tramite DiPrint e Anti-Plugin]{Risultati di identificazione tramite DiPrint e Anti-Plugin per applicazione\\ false se l'ambiente rilevato è nativo, true se l'ambiente rilevato è virtualizzato \\ A1 = com.lbe.parallel.intl \\ A2 = com.ludashi.dualspace \\ A3 = 
info.cloneapp.mochat.in.goast \\ A4 = 
com.parallel.space.lite \\ A5 =
com.exelliance.multiaccounts \\ A6 =
com.ludashi.superboost \\ A7 =
com.in.parallel.accounts \\ A8 =
com.polestar.domultiple}
\label{tab:antidiprint_device_tests}


\end{table}