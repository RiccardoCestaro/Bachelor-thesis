%!TEX root = ../dissertation.tex
\chapter{Conclusione}
\label{conclusion}

Gli ambienti virtualizzati in \emph{Android} nascondono molti problemi di sicurezza che attirano molti sviluppatori con scopi sospetti a sfruttarli a loro vantaggio. 
Gli \emph{\gls{hookg}} applicati a \emph{Màscara} per aggirare le librerie di identificazione attualmente esistenti, come \emph{DiPrint} e \emph{Anti-Plugin}, hanno dimostrato la facilità di creazione di un malware che possa inviare dati sensibili a un server esterno, senza che l'utente ne sia a conoscenza.

\emph{Singular}, la libreria creata durante il periodo di stage, è la prima per l'individuazione automatica di ambienti virtualizzati che si basa sulle strutture interne dell'\emph{\gls{artg}} accedendo ai suoi valori direttamente dalla memoria centrale in modo da mitigare la possibilità di applicare degli \emph{\gls{hookg}} cambiando i valori di ritorno delle \emph{API} o funzioni.
In particolare, \emph{Singular} controlla il valore dell'\emph{hotness\_count} al primo avvio di una applicazione, in modo da individuare un ambiente virtualizzato nel caso in cui i metodi della classi \emph{ActivityThread} siano compilati \emph{AoT} e non \emph{JiT}.
Il metodo implementato in \emph{Singular} risulta essere più robusto rispetto alle tecniche utilizzate dalle uniche due librerie attualmente esistenti ma non è negata l'esistenza di possibili contromisure.

