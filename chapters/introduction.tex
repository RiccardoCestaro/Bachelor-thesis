%!TEX root = ../dissertation.tex
\chapter{Introduzione}
\label{introduction}


\emph{Android} è uno dei sistemi operativi più diffusi e ogni giorno vengono pubblicate applicazioni di ogni genere nel \emph{\gls{playstoreg}}\glsfirstoccur, senza che venga applicata una selezione accurata. Infatti molte applicazioni, create da sviluppatori con scopi sospetti, sono disponibili al download a tutti tramite un tap. Un utente, ignaro dall'inganno, potrebbe scaricare applicazioni ed essere vittima di \emph{\gls{phishingg}}\glsfirstoccurspace  senza che se ne accorga. In particolare, negli ultimi anni, gli attaccanti hanno scoperto la strada della virtualizzazione per la creazione di malware, ideata inizialmente per poter superare le dimensioni limite dei pacchetti e il limite sul numero di metodi implementabili, originariamente fermo a \emph{65535}.
Sono molte le applicazioni che utilizzano la virtualizzazione e le più comuni permettono agli utenti di duplicare le loro applicazioni, in modo da permettere di utilizzare più account di un servizio contemporaneamente. Infatti è una pratica comune avere due account di \emph{Whatsapp} o \emph{Facebook} e semplificarne la gestione tramite la virtualizzazione.


I problemi di sicurezza legati agli ambienti di virtualizzazione in \emph{Android} sono molteplici e sempre più malware sono accusati di sfruttare questo sistema. Un'applicazione inizialmente innocua scaricata dal \emph{\gls{playstoreg}} potrebbe scaricare un altro pacchetto a runtime ed avviarlo tramite questo sistema. Il pacchetto scaricato e virtualizzato potrebbe ottenere tutti i permessi dell'applicazione in cui è eseguita accedendo a dati personali come contatti, foto o qualsiasi cosa condivisa con l'applicazione nota all'utente. In un altro modo, un utente potrebbe installare un pacchetto creato da terze parti allo scopo di aggiungere funzionalità ad applicazioni note, ma che in realtà nasconde altri comportamente malevoli. Infatti potrebbe avviare l'applicazione originale all'interno di un ambiente virtualizzato riuscendo ad accedere ed esportare dati sensibili come dati di accesso.
Grazie al \emph{Kaspersky Lab analysis} è stata confermata la presenza di 870,617 applicazioni malevoli presenti sul \emph{\gls{playstoreg}} nel 2019 \cite{KasperskyMalwareNumber}.

L'inefficacia delle librerie di identificazione di ambienti virtualizzati attualmente esistenti, \emph{DiPrint} \cite{DiPrint} e \emph{Anti-Plugin} \cite{Antiplugin}, verrà dimostrata aggiungendo del codice malevolo al malware \emph{Màscara} che permetta di intercettare e modificare le chiamate alle \emph{\gls{apig}}\glsfirstoccurspace utilizzate dalle suddette librerie.
Il grande difetto di \emph{DiPrint} e \emph{Anti-Plugin} è il livello in cui operano. Infatti, implementano diverse tecniche per l'identificazione di ambienti virtualizzati che utilizzano le \emph{\gls{apig}} di \emph{Java} o del \emph{framework di Android}, e quindi ad alto livello.

Un metodo di detection più efficace e robusto, che sfrutta le informazioni ricavate dalle strutture del \emph{\gls{artg}}\glsfirstoccur, è stato concretizzato e reso disponibile attraverso la libreria \emph{Singular}. 
\emph{Singular} permette di identificare un ambiente virtualizzato verificando la complessità dei metodi utilizzati per l'avvio delle applicazioni all' interno dell' ambiente. La libreria ha un rate di successo pari al 100\% anche se la sua compatibilità si ferma al 60.8\%\footnote{Dati aggiornati a Maggio 2020} dei dispositivi, con la possibilità di essere utilizzata da \emph{Android Oreo 8.0} ad \emph{Android 10}.
I test effettuati dimostrano che la tecnica implementata in \emph{Singular} è più robusta e sofisticata rispetto alle tecniche di identificazione implementate nelle librerie \emph{DiPrint} e \emph{Anti-Plugin}.

\newpage
\section{Struttura del documento}

\begin{enumerate}

\item \nameref{introduction}: descrizione generale e struttura del documento.
\item \nameref{chap:art}: descrizione del \emph{\gls{artg}}, evidenziando maggiormente il nuovo tipo di compilazione e la struttura di un \emph{\gls{artmethodg}}.
\item \nameref{chap:virt_env}: descrizione degli ambienti virtualizzati in \emph{Android} mostrando la facilità di creazione di un malware, attraverso \emph{Màscara}.

\item \nameref{chap:lib_esis}: descrizione e analisi delle librerie attualmente esistenti per l'identificazione di ambienti virtualizzati, spiegando come è possibile aggirare le tecniche di detection attualmente esistenti.
\item \nameref{chap:analisi_art}: analisi delle strutture interne del \emph{\gls{artg}} per lo studio di possibili tecniche di identificazione di ambienti virtualizzati.
\item \nameref{chap:singular}: implementazione di \emph{Singular}, mostrando la sua efficacia, la sua compatibilità e delle possibili contromisure.
\item \nameref{conclusion}: considerazioni finali.
\end{enumerate}